\section*{Úkol 2}
\label{sec:task-2}

Porovnejte nárůsty ve výkonnostních skórech (FPS) pro verzi hry "Cyberpunk 2077" po aplikaci 1.5 patche (dále jen \textbf{„nárůst FPS“}) 
pro vybrané grafické karty. Nezapomeňte, že použité metody mohou vyžadovat splnění určitých předpokladů. Pokud tomu tak bude, okomentujte 
splnění/nesplnění těchto předpokladů jak na základě explorační analýzy (např. s odkazem na histogram apod.), tak exaktně pomocí metod statistické indukce.

\newcommand{\lowerThanZero}{$<$$<$ 0.001}

\newcommand{\shapireWilk}{0.006, 0.019}
\newcommand{\symmetryTest}{0.218, 0.016}
\newcommand{\ttValues}{5.600, "inf", 5.100, "inf"}
\newcommand{\rtxInterval}{(\tableValue{\ttValues}{0}, \tableValue{\ttValues}{1}) }
\newcommand{\amdInterval}{(\tableValue{\ttValues}{2}, \tableValue{\ttValues}{3}) }

\newcommand{\pointGuess}{0.400}
\newcommand{\wilcoxLeftSidedInterval}{$(0.399, \infty)$}

\begin{enumerate}[label=\alph*)]
    \item Graficky prezentujte srovnání nárůstu FPS pro grafické karty \nvidiaCardTri\ a \amdCardSedm\ (vícenásobný krabicový graf, histogramy, q-q grafy).
    Srovnání okomentujte (včetně informace o případné manipulaci s datovým souborem).

    \vspace{1em}
    \begin{minipage}{0.94\textwidth}
        Pro srovnání nárůstu FPS pro jednotlivé grafické karty je možné využít krabicový graf na \figref{fig:box_plot}, histogramy na \figref{fig:histogram} a q-q grafy na \figref{fig:qq}.
        V těchto grafech již nejsou zobrazeny odlehlé hodnoty, které byly identifikovány v úkolu 1.
    \end{minipage}

    \newpage
    \item Na hladině významnosti 5 \% rozhodněte, zda jsou střední hodnoty nárůstů FPS (popř.\ mediány nárustů FPS) pro grafické karty \nvidiaCardTri\ a
    \amdCardSedm\ statisticky významné.\ K řešení využijte bodové a intervalové odhady i čistý test významnosti.\ Výsledky okomentujte.

    \vspace{1em}
    \begin{minipage}{0.94\textwidth}
        Na základě prezentovaných Q-Q grafů (viz \figref{fig:qq}) lze usuzovat, že nárůst FPS u obou grafických karet odpovídá normálnímu rozdělení.
        Toto tvrzení podporují i hodnoty výběrové šikmosti a špičatosti, které spadají do intervalu (-2, 2) (viz \tabref{tab:characteristics-summary}). \\

        Dle Shapirova-Wilkova testu (viz \tabref{tab:normality-test}) nelze na hladině významnosti 0.05 nárůst FPS obou typů grafických karet považovat za normálně rozdělený.
        Z toho důvodu je nutné použít neparametrické testy pro ověření významnosti mediánu nárůstu FPS (viz \tabref{tab:deterministic-normality-test}).

        \captionof{table}{Ověření normality nárůstu FPS dle grafických karet}
        \label{tab:normality-test}
        \renewcommand{\arraystretch}{1.3}
        \resizebox{\textwidth}{!}{%
            \begin{tabular}{|p{5cm}|P{3.5cm}|P{3.5cm}|P{8cm}|}
                \hline
                            & \textbf{šikmost}                & \textbf{špičatost}              & \textbf{Shapirův-Wilkův test (p-hodnota)} \\ \hline
                \nvidiaCardTri & \tableValue{\skewnessValues}{2} & \tableValue{\kurtosisValues}{2} & \tableValue{\shapireWilk}{0}              \\ \hline
                \amdCardSedm    & \tableValue{\skewnessValues}{3} & \tableValue{\kurtosisValues}{3} & \tableValue{\shapireWilk}{1}              \\ \hline
            \end{tabular}%
        }
        \vspace{1em}

        % TODO: Barče, použít u obou Wilcoxon??
        Na základě testu symetrie (viz \tabref{tab:deterministic-normality-test}) bylo u grafické karty \nvidiaCardTri\ nebylo zamítnuto symetrické rozdělení, zatímco u grafické karty \amdCardSedm\ bylo.\@
        Vzhledem k tomu, že výsledky scheme srovnávat tak budeme dále předpokládat, že nárůst FPS u obou grafických karet není symetrický.
        Po provedení SIGN testu bylo zjištěno, že medián nárůstu FPS je pro obě grafické karty významně vyšší než 0.
        Lze tedy konstatovat, že medián nárůstu FPS je pro obě grafické karty statisticky významný.

        \captionof{table}{Podrobné ověření normality nárůstu FPS dle grafických karet}
        \label{tab:deterministic-normality-test}
        \vspace{0.5em}
        \renewcommand{\arraystretch}{1.3}
        \resizebox{\textwidth}{!}{%
            \begin{tabular}{|p{5cm}|P{5cm}|P{10cm}|}
                \hline
                            & \textbf{Test symetrie}        & \textbf{SIGN test} \\ \hline
                \nvidiaCardTri & \tableValue{\symmetryTest}{0} & \lowerThanZero     \\ \hline
                \amdCardSedm    & \tableValue{\symmetryTest}{1} & \lowerThanZero     \\ \hline
            \end{tabular}%
        }
        \vspace{1em}

        Vzhledem k tomu, že očekáváme kladné hodnoty nárůstu FPS (výkonnostní skóre se zvyšuje), volíme levostranné intervalové odhady a levostranné testy.
        U grafické karty \nvidiaCardTri\ lze očekávat, že průměrný nárůst FPS bude přibližně \tableValue{\meanValues}{2}.\@
        95\% levostranný intervalový odhad průměrného nárůstu FPS pro tuto kartu je \rtxInterval FPS.\@
        Intervalový odhad i p-hodnota levostranného t-testu ukazují, že průměrný nárůst FPS je statisticky významně větší než nula.
        Jinými slovy, na hladině významnosti 0,05 lze nárůst FPS pro grafickou kartu \nvidiaCardTri\ považovat za statisticky významný. \\

        U grafické karty \amdCardSedm\ lze odhady interpretovat obdobně.
        Nárůst FPS pro tuto kartu je přibližně \tableValue{\meanValues}{3}.\@ a 95\% levostranný intervalový odhad průměrného nárůstu FPS je \amdInterval FPS.\@

        \captionof{table}{Bodové a intervalové odhady střední hodnoty nárůstu FPS dle grafických karet}
        \label{tab:interval-estimation}
        \vspace{0.5em}
        \renewcommand{\arraystretch}{1.3}
        \resizebox{\textwidth}{!}{%
            \begin{tabular}{|p{5cm}|P{4cm}|P{7cm}|P{4cm}|}
                \hline
                            & \textbf{bodový odhad (FPS)} & \textbf{95\% levostranný intervalový odhad (FPS)} & \textbf{Levostranný t-test (p-hodnota)} \\ \hline
                \nvidiaCardTri & \tableValue{\meanValues}{2} & \rtxInterval                                      & \lowerThanZero                          \\ \hline
                \amdCardSedm    & \tableValue{\meanValues}{3} & \amdInterval                                      & \lowerThanZero                          \\ \hline
            \end{tabular}%
        }
    \end{minipage}

    \newpage
    \item Pro grafické karty \nvidiaCardTri\ a \amdCardSedm\ rozhodněte (na hladině významnosti 5 \%), zda se jejich střední hodnoty (popř.\ mediány)
    nárůstu FPS po aplikaci 1.5 patche statisticky významně liší.
    K řešení využijte bodový a intervalový odhad i čistý test významnosti.
    Výsledky okomentujte.

    \vspace{1em}
    \begin{minipage}{0.94\textwidth}
        Vzhledem k zamítnutí předpokladu normality nárůstu FPS (viz \tabref{tab:normality-test}) budeme i nadále pokračovat v aplikaci neparametrických metod. \\

        Tvar rozdělení nárůstu FPS u obou grafických karet je srovnatelný (viz \figref{fig:histogram}).
        Pro odhad a test významnosti rozdílu mediánů nárůstu FPS využijeme metody založené na Mannově-Whitneyho statistice.

        \captionof{table}{Srovnání mediánů nárustu FPS grafické karty \nvidiaCardTri\ a \amdCardSedm)}
        \label{tab:median-comparison}
        \vspace{0.5em}
        \renewcommand{\arraystretch}{1.3}
        \resizebox{\textwidth}{!}{%
            \begin{tabular}{p{12cm}P{3cm}}
                \hline
                \textbf{Bodový odhad $x^{Nvidia}_{0.5} - x^{AMD}_{0.5}$ (mAh)}                         & \pointGuess              \\
                \textbf{95\% levostranný intervalový odhad $x^{Nvidia}_{0.5} - x^{AMD}_{0.5}$ (mAh)}   & \wilcoxLeftSidedInterval \\
                \textbf{Mannův-Whitneyho levostranný test (p-hodnota)}                                 & \lowerThanZero           \\
                \hline
            \end{tabular}%
        }
        \vspace{1em}

        U grafické karty \nvidiaCardTri\ lze očekávat, že medián nárůstu FPS bude přibližně o \pointGuess\ větší než u grafické karty \amdCardSedm.\@
        Odpovídající 95\% levostranný intervalový odhad rozdílu mediánů je \wilcoxLeftSidedInterval.\@
        Mannův-Whitneyho levostranný test ukazuje, že nárůst FPS u grafické karty \nvidiaCardTri\ je na hladině významnosti 5\% statisticky významně
        vyšší než u grafické karty \amdCardSedm.\@
    \end{minipage}
\end{enumerate}

\endinput