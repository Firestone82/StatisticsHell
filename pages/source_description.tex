\section*{Popis datového souboru}
\label{sec:data-source-description}

V datovém souboru jsou zaznamenány výkonnostní skóre (FPS) čtyř populárních grafických karet:
Nvidia RTX 2080 Ti, Nvidia RTX 3070 Ti, AMD Radeon RX 6800 XT a AMD Radeon RX 7700 XT. Tyto karty
byly testovány ve hře "Cyberpunk 2077" ve dvou různých verzích: původní release a po aplikaci 1.5
patche. Vaším úkolem je analyzovat, jak patch 1.5 ovlivnil výkonnostní skóre těchto karet ve hře. Pro
každý unikátní testovaný systém (test) viz (id) byly testovány obě verze hry

\vspace{1em}
\noindent
V souboru ukol\_X.xlsx jsou pro každý test uvedeny následující údaje:

\begin{itemize}
  \item \textbf{id} ... identifikátor testovaného systému (každý systém je unikátní PC systém)
  \item \textbf{typ karty} ... Nvidia RTX 2080 Ti, Nvidia RTX 3070 Ti, AMD Radeon RX 6800 XT, AMD Radeon RX 7700 XT
  \item \textbf{testovaná verze} ... „release“ a „patched“
  \item \textbf{FPS} ... naměřené výkonnostní skóre (FPS) pro danou grafickou kartu s danou verzi hry.
\end{itemize}

\noindent
Obecné pokyny:

\begin{itemize}
    \item Domácí úkoly odevzdávejte vždy v termínu, který určil váš cvičící.
    \item Portfolio domácích úkolů budete odevzdávat postupně. Tj. nejdříve odevzdáte titulní stránku s úkolem 1, k okomentovanému úkolu 1 připojíte úkol 2 atd
    \item Domácí úkoly zpracujte dle obecně známých typografických pravidel. 
    \item Všechny tabulky i obrázky musí být opatřeny titulkem, který obsahuje i očíslování objektu.  
    \item Do domácích úkolů nevkládejte tabulky a obrázky, na něž se v doprovodném textu nebudete odkazovat.
    \item Bude-li to potřeba, citujte zdroje dle mezinárodně platné citační normy ČSN ISO 690.
\end{itemize}

\endinput